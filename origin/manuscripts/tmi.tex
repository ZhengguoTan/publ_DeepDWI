\documentclass[journal,twoside,web]{ieeecolor}
\usepackage{tmi}
\usepackage{cite}
\usepackage{amsmath,amssymb,amsfonts}
\usepackage{algorithmic}
\usepackage{graphicx}
\usepackage{textcomp}
\usepackage{hyperref}
\usepackage[capitalise,noabbrev]{cleveref}
\def\BibTeX{{\rm B\kern-.05em{\sc i\kern-.025em b}\kern-.08em
	T\kern-.1667em\lower.7ex\hbox{E}\kern-.125emX}}
\markboth{\journalname, VOL. XX, NO. XX, XXXX 2024}
{Tan \MakeLowercase{\textit{et al.}}: DeepDWI}

\begin{document}
	\title{Diffusion-Weighted Imaging with Learned Nonlinear Latent Space Modeling and Self-Supervised Reconstruction (DeepDWI)}

	\author{Zhengguo Tan, Julius Glaser, Patrick A Liebig, Annika Hofmann, Frederik B Laun, Florian Knoll
		\thanks{This work was supported in part by
			German Research Foundation (DFG)
			under projects 513220538 and 512819079,
			project 500888779 in the Research Unit RU5534
			for MR biosignatures at UHF,
			and by the National Institutes of Health (NIH)
			under grants R01 EB024532 and P41 EB017183.
			In addition, scientific support and HPC resources
			were provided by
			the Erlangen National High Performance Computing Center (NHR)
			of Friedrich-Alexander-University Erlangen-Nuremberg (FAU)
			under the NHR project b143dc.
			NHR is funded by federal and Bavarian state authorities.
			NHR@FAU hardware is partially funded by
			DFG under project 440719683.}
		\thanks{Z.~T. was with the Department
			Artificial Intelligence in Biomedical Engineering (AIBE),
			FAU, Erlangen, Germany.
			He is now with
			the Michigan Institute for Imaging Technology and Translation
			(MIITT),
			Department of Radiology,
			University of Michigan, Ann Arbor, MI 48109 USA
			(e-mail: zgtan@med.umich.edu).}
		\thanks{J.~G. is with the Department Medical Engineering,
			FAU, Erlangen, Germany
			(e-mail: julius.glaser@fau.de).}
		\thanks{P.~A.~L. is with Siemens Healthcare GmbH, Erlangen, Germany
			(e-mail: patrick.liebig@siemens-healthineers.com).}
		\thanks{A.~H. is with the Department AIBE,
			FAU, Erlangen, Germany
			(e-mail: annika.ah.hofmann@fau.de).}
		\thanks{F.~B.~L. is with the Institute of Radiology,
			University Hospital Erlangen,
			FAU, Erlangen, Germany
			(e-mail: Frederik.Laun@uk-erlangen.de).}
		\thanks{F.~K. is with the Department AIBE,
			FAU, Erlangen, Germany
			(e-mail: florian.knoll@fau.de).}
	}

	\maketitle

	% 250 words
	\begin{abstract}
		% These instructions provide guidelines for preparing papers for IEEE Transactions,
		% but this version is specifically written to describe submission to IEEE TMI.
		% Use this document as a template if you are using \LaTeX.
		% Otherwise, use this document as an instruction set.
		% The electronic file of your paper will be formatted further at IEEE.
		% Paper titles should be written in uppercase and lowercase letters, not all uppercase.
		% Avoid writing long formulas with subscripts in the title;
		% short formulas that identify the elements are fine (e.g., "Nd--Fe--B").
		% Keep the title short and do not write ``(Invited)'' in the title.
		% Full names of authors are preferred in the author field, but are not required.
		% Put a space between authors' initials. Only authors may appear in the author line
		% of a manuscript. Authors are defined as individuals who have made an identifiable
		% intellectual contribution to a manuscript to the extent that the individual can defend its contents.
		% Define all symbols used in the abstract. Do not cite references in the abstract.
		Keep the abstract to 250 words or less.
	\end{abstract}

	\begin{IEEEkeywords}
	Diffusion-weighted imaging, Image reconstruction, Neural network, Latent space, Self-supervised learning
	\end{IEEEkeywords}

	% ============================== %
	\section{Introduction}
	\label{SEC:INTRO}
	\IEEEPARstart{H}{igh}-dimensional magnetic resonance imaging (HD-MRI),
	referring to the acquisition, reconstruction, and analysis of
	multi-dimensional imaging,
    in contrast to single-contrast-weighted, static, and two-dimensional imaging.
	Examples of HD-MRI include but are not limited to
    magnetic resonance spectroscopic imaging (MRSI) \cite{brown_1982_mrsi},
    diffusion-weighted imaging (DWI) \cite{lebihan_1986_diff},
    and magnetic resonance fingerprinting (MRF) \cite{ma_2013_mrf}.
    MRSI uses multiple readout gradients to acquire multiple echo images
    for the computation of spatially resolved metabolic distribution.
    DWI utilizes spatially and angularly varying
    diffusion encoding gradients
    to obtain multi-contrast diffusion-weighted images
    as a probe into tissue microstructure.
    MRF designs a $T_1$- and $T_2$-prepared pseudo-randomized sequence
    to acquire time-resolved transient-state images,
    which are matched with Bloch-equation generated dictionaries \cite{doneva_2010_moba}
	for simultaneous quantitative $T_1$ and $T_2$ mapping.

	HD-MRI, however, conventionally requires long scan time
	and high computational burden.
	Advances in parallel imaging
	\cite{roemer_1990_pi,sodickson_1997_smash,
	pruessmann_1999_sense,pruessmann_2001_gsense,griswold_2002_grappa}
	and compressed sensing
	\cite{lustig_2007_cs,block_2007_cs,liang_2007_psf}
	have enabled accelerated acquisition for HD-MRI. For instance,
	Lam et al.~\cite{lam_2014_spice} proposed SPectroscopic Imaging
	by exploiting spatiospectral CorrElation (SPICE)
	based on the low-rank modeling.
	% Block et al.~\cite{block_2014_rad} and Feng et al.~\cite{feng_2014_grasp}
	% proposed iterative golden-angle radial sparse parallel imaging (GRASP)
	% with total variation regularization \cite{rudin_1992_tv} in the time domain,
	% enabling robust free-breathing time-resolved abdominal imaging and
	% breast dynamic contrast-enhanced imaging with reduced motion artifacts.
	McGivney et al.~\cite{mcgivney_2014_svdmrf}
	proposed the use of the singular value decomposition (SVD) and thresholding
	to compress the MRF dictionary and to reduce the computational burden.
	Further, Christodoulou et al.~\cite{christodoulou_2018_mt}
	proposed MR Multitasking with higher-order SVD (HOSVD) modeling
	and iterative reconstruction
	for motion-resolved quantitative $T_1$ and $T_2$ mapping.
    However, the use of patch-based SVD still requires long computational time.

	Beyond sparsity constraint and low-rank modeling,
	advanced neural networks, e.g.~denoising autoencoder \cite{hinton_2006_ae},
	have been explored for HD-MRI reconstruction.
	Lam et al.~\cite{lam_2019_mrsi} proposed
	to first learn a DAE model
	from physics-informed simulated data
	and then incorporate the learned DAE model as a regularizer
	in iterative reconstruction.
	This concept was adopted by Mani et al.~\cite{mani_2021_qmodel}
	for joint $k$-$q$-space DWI reconstruction using learned DAE \textit{priors}.
	Further, Arefeen et al.~\cite{arefeen_2023_latent} proposed
	to replace the conventional SVD-based linear subspace modeling
    \cite{huang_2012_t2basis}
	by the latent decoder model within DAE
	for improved multi-$T_2$-weighted image reconstruction.
	Besides learning a \textit{prior} based on simulated data
	for regularization or latent space modeling,
	Hammernik et al.~\cite{hammernik_2018_varnet} and
	Aggarwal et al.~\cite{aggarwal_2018_modl}
	proposed supervised learning unroll reconstruction networks,
	which are trained by fully-sampled in vivo data.
	Yaman et al.~\cite{yaman_2020_ssdu,yaman_2022_zs}
	proposed the self-supervised learning unroll network
    without fully-sampled data,
	which builds upon the concept of cross-validation in machine learning.

    The capability of DAE to learn diffusion MRI models, however, 



	% ============================== %
	\section{Theory}


	% ============================== %
	\section{Methods}

    \subsection{Learning a VAE}


	% ============================== %
	\section{Results}


	% ============================== %
	\section{Discussion}


	% ============================== %
	\section{Conclusion}


	% ============================== %
	\section*{Acknowledgment}

	Z.~T. thanks to Ms.~Soundarya Soundarresan for
	her work and discussion on denoising autoencoder.
	Z.~T. thanks to Dr.~Xiaoqing Wang for
	the discussion on self-supervised learning.

	% ============================== %
	\bibliographystyle{IEEEtran}
	\bibliography{../../ref/ref}

\end{document}
